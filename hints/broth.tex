
\section{Broth and Stock}
{\footnotesize From: \url{https://www.eater.com/23552129/what-is-in-store-bought-chicken-stock}}

Culinarily, a ``stock'' is something cooked in water (e.g., vegetables or bones).
``Broth'' is a stock with salt or spices added. Stock is an ingredient, broth is a
food.

The FDA does not define stock or broth other than to require a ``Moisture Protein
Ratio (MPR) of 135:1 or below'' (\qty{1.68}{g} protein per liquid cup). Therefore at
the supermarket, there is no difference between stock, broth, or bone broth.

In practice, store-bought broths are (of course) heavy chemistry. Broth manufacturers
purchase flavor
reductions\footnote{\url{https://essentiaproteins.com/na/ingredients/food-savory/stocks-broths-and-fats/}}
then add water, fats, additional proteins, additional flavors, and salt to make
something that they can call a broth. Extrapolating, the companies producing the
flavor reductions boil the carcasses and split the flavoring from the gelatin and the
fat in order to sell each separately. The industrial reductions \textsl{are} high in
protein so it is still unclear what protein exactly they contain that isn't gelatin
-- my guess is that the reduction only hydrolizes some of the proteins into gelatin
which are removed and the remainders are mostly collagen. The broth companies can add
any form of protein which is why you can get something that doesn't gelatinize --
presumably some market research at some point told them that customers don't want
blobs coming out of the can/box and/or they couldn't keep the product consistent
enough. In any case, store-bought broths are very much a processed food which has
been deconstructed and reassembled for the sake of ease of processing, consistency,
cost, and brand specialization.

Note: In order to improve the texture of a store-bought broth, one could add a packet
of unflavored gelatin to 3--4~cups of store-bought broth before cooking.


%%% Local Variables:
%%% mode: latex
%%% TeX-master: "../Cookbook"
%%% End:
