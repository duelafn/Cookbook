
\section{Conversions}

\subsection{Volume}
\begin{center}
  \begin{tabular}{l*8{c}}\toprule
           & tsp   & Tbsp     & fl oz    & cup         & pint        & quart        & liter  & gallon\\\midrule
    tsp    & --    & \vfrac13 & \vfrac16 & \vfrac1{48} & \vfrac1{96} & \vfrac1{192} & 0.0049 & \vfrac1{768}\\
    Tbsp   & 3     & --       & \vfrac12 & \vfrac1{16} & \vfrac1{32} & \vfrac1{64}  & 0.0148 & \vfrac1{256}\\
    fl oz  & 6     & 2        & --       & \vfrac18    & \vfrac1{16} & \vfrac1{32}  & 0.0296 & \vfrac1{128}\\
    cup    & 48    & 16       & 8        & --          & \vfrac12    & \vfrac14     & 0.2366 & \vfrac1{16}\\
    pint   & 96    & 32       & 16       & 2           & --          & \vfrac12     & 0.4732 & \vfrac18\\
    quart  & 192   & 64       & 32       & 4           & 2           & --           & 0.9464 & \vfrac14\\
    liter  & 202.9 & 67.63    & 33.81    & 4.227       & 2.113       & 1.057        & --     & 0.2642\\
    gallon & 768   & 256      & 128      & 16          & 8           & 4            & 3.7854 & --\\\bottomrule
  \end{tabular}
\end{center}

1 pinch = about 1/16 teaspoon = the amount you can hold between your thumb and two fingers


\subsection{Weight}
\begin{center}
  \begin{tabular}{l*4{c}}\toprule
          & gram   & ounce  & pound  & kg\\\midrule
    gram  & --     & 0.0353 & 0.0022 & 0.0010\\
    ounce & 28.35  & --     & 0.0625 & 0.0283\\
    pound & 453.59 & 16     & --     & 0.4536\\
    kg    & 1000   & 35.27  & 2.20   & --\\\bottomrule
  \end{tabular}
\end{center}

\subsection{Volume $\leftrightarrow$ Weight}

\begin{center}
\begin{longtable}{lrr}\toprule
  \textbf{Ingredient}   & \textbf{Volume}      & \textbf{Weight}\\
  \midrule\endhead
  \bottomrule\endfoot
  almonds               & 1.5 cups (shell off) & 1 lb (shell on)\\
  cheese (shredded)     & 1 cup                & 4 oz\\
  chocolate (chips)     & 1 cup                & 6 oz\\
  chocolate (grated)    & \fourth cup          & 1 oz\\
  cocoa                 & 1 cup                & \fourth lb\\
  fat, pure (lard)      & 1 cup                & \half lb\\
  flour, cake unsifted  & 1 cup                & 115 grams\\
  flour, white packed   & 1 cup                & 180 grams\\
  flour, white sifted   & 1 cup                & 123 grams\\
  flour, white unsifted & 1 cup                & 150 grams\\
  honey                 & 1 cup                & 325 grams\\
  pecans                & 2.5 cups (shell off) & 1 lb (shell on)\\
  rice, white, dry      & 1 cup                & \half lb\\
  salt, table           & 1 cup                & 200 grams\\
  sugar, brown packed   & 1 cup                & 200 grams\\
  sugar, white          & 1 cup                & 200 grams\\
  tomato                & 1 cup (chopped)      & \twothird lb\\
  walnuts               & 2 cups (shell off)   & 1 lb (shell on)\\
  water                 & 1 cup                & 236.7 grams\\
\end{longtable}
\end{center}

\subsection{Other Item Conversions}

% https://www.epicurious.com/expert-advice/spice-conversion-whole-to-ground-article

\begin{center}
\def\TwoCol#1{\multicolumn{2}{r}{#1}}
\def\SORT#1{\xspace}
\begin{longtable}{lr@{\quad=\quad}r}
  \toprule\endhead
  \bottomrule\endfoot
  apple                            & 1 medium              & 1 cup chopped\\
  apple                            & 1 medium              & 1 lb\\
  bread crumbs                     & 1 slice               & \half cup crumbs\\
  butter                           & 1 stick               & \fourth lb\\
  butter                           & 1 stick               & \half cup\\
  cream,\SORT{1} half-and-half     & \TwoCol{11--18\% butterfat}\\
  cream,\SORT{2} light             & \TwoCol{18\% butterfat}\\
  cream,\SORT{3} light whipping    & \TwoCol{26--30\% butterfat}\\
  cream,\SORT{4} heavy whipping    & \TwoCol{36\% or more butterfat}\\
  cream,\SORT{5} double or clotted & \TwoCol{42\% butterfat}\\
  cumin                            & 2 \t seeds            & 1 \t powder\\
  egg                              & 1 cup                 & 4--6 large\\
  egg                              & 1 large               & 3 \T\\
  egg, white                       & 1 large               & 2 \T\\
  egg, yolk                        & 1 large               & 1 \T\\
  garlic                           & med. clove            & 1 \t\\
  gelatin                          & 1 packet              & 1 \T\\
  herbs                            & 1 \T dried            & 3 \T fresh\\
  honey                            & 1 stick               & \threefourth \t\\
  lemon / lime                     & 1 medium              & 2 \T juice\\
  mustard                          & 1 \T                  & 1 \t dry\\
  onion                            & 1 large               & 1 cup chopped\\
  orange                           & 1 medium              & \fourth cup juice\\
  pasta, macaroni                  & 1 cup dry             & 3 cups cooked\\
%   pasta, spaghetti               & 1\threefourth cup dry & 4--5 cups cooked\\
  popcorn                          & \fourth cup kernels   & 5 cups popped\\
  rice (white)                     & 1 cup dry             & 3 cups cooked\\
  yeast                            & 1 packet              & 2\fourth \t
\end{longtable}
\end{center}


% \item{Ground Beef}
% Ground chuck usually is 15\% fat\\
% Ground round usually is about 10\% fat\\
% Ground sirloin usually is about 5\% fat\\
% 1 pound boneless meat = 3 cups cubed meat
%
% \item{Pasta, rice}
% 1 cup uncooked white rice = 1/2 pound\\
% 1 cup uncooked white rice + 2 cups boiling water = 3 cups cooked\\
% 1 cup brown whole grain rice = 1/2 pound + 2.25 cups boiling water = 4 cups of cooked rice
%
% \item{Salt}
% 1 teaspoon table salt =
% 1.5 teaspoons Kosher Salt =
% 3 to 4 or more teaspoons of sea salt
%
% \item{Sugars}
% 1 cup packed dark brown sugar = 6 ounces by weight = 250 grams by weight\\
% 1 cup packed dark brown sugar = 1 cup white sugar + 2 tablespoons molasses\\
% 1 cup packed dark brown sugar = 1 cup of light brown sugar + 1 tablespoon molasses\\
% 1 cup honey = 3/4 cup sugar + 1/4 cup water\\
% 1 cup corn syrup = 1 cup sugar dissolved in 1/4 cup water\\
% Simple syrup is 1 cup of sugar with 1 cup of water thoroughly dissolved
%
% \item{Other}


% Salt is 2.16 x density of water
%
% Foolproof Basic Brine. Add one cup of hot water to a two cup measuring cup.
% Then pour in salt until the water line reaches 1.5 cups. Produces a 6.3\%
% brine.
%
% 1/2" thick meat should be submerged in brine for 1/2 hour in the refrigerator
% 1" thick meat should be submerged in brine for 1 hour in the refrigerator
% 2" thick meat should be submerged in brine for 3 hours in the refrigerator
% 3" thick meat should be submerged in brine for 8 hours in the refrigerator
%
% Celsius or Centigrade = (Fahrenheit - 32) / 1.8
% Fahrenheit = (Celsius x 1.8) + 32C = (F - 32) / 1.8
%
% 400°F = 204C = hot oven
% 350°F = 177C
% 300°F = 149C
% 250°F = 121C
% 225°F = 107C = ideal smoke roasting temperature
% 212F = 100C = water boils
% 180°F = 82C
% 170°F = 77C
% 160°F = 71C
% 68F = 20C = room temperature
% 32F = 0C = water freezes
% 0°F = -18C
%
% Boiling point goes down about 2°F for every 1000 feet above sea level.
%
%
%
% % http://www.fareshare.net/conversions-volume-to-weight.html
% Item 	Volume 	Ounces
% Allspice, ground 	Tablespoon 	1/4
% Almonds, blanched 	Cup 	5 1/3
% Apples, peeled, 1/2" cubes 	Cup 	3 1/3
% Applesauce, canned 	Cup 	8
% Apples, pie, canned 	Cup 	6
% Apricots, drained 	Cup 	5 1/3
% Apricots, cooked 	Cup 	3 1/3
% Apricots, halves 	Cup 	8
% Apricots, pie, packed 	Cup 	9
% Asparagus, cut, canned 	Cup 	6 1/2
%
%
% Item Volume 	Ounces  	Food Index
% Baking powder 	Tablespoon 	1/2
% Baking powder 	Cup 	8
% Bananas, diced 	Cup 	6 1/2
% Barley 	Cup 	8
% Beans, bakes 	Cup 	8
% Beans, lima, dried 	Cup 	6 1/2
% Beans, lima, cooked 	Cup 	8
% Beans, kidney 	Cup 	6
% Beans, kidney, cooked 	Cup 	6 3/4
% Beans, navy, dried 	Cup 	6 3/4
% Beans, navy, cooked 	Cup 	5 1/3
% Beans, cut, canned, drained 	Cup 	4 1/2
% Bean sprouts 	Cup 	4
% Beets, cooked, diced 	Cup 	6 1/2
% Beets, cooked, sliced 	Cup 	6 1/2
% Blueberries, fresh 	Cup 	7
% Blueberries, canned 	Cup 	6 1/2
% Bread crumbs, dried 	Cup 	4
% Bread crumbs, soft 	Cup 	2
% Brussels sprouts 	Cup 	4
% Butter 	Cup 	8
%
%
% Item Volume 	 Ounces  	Food Index
% Cabbage, shredded 	Cup 	4
% Cake crumbs, soft 	Cup 	2 3/4
% Carrots, diced, raw or cooked 	Cup 	5 1/3
% Celery, diced 	Cup 	4
% Celery seed 	Tablespoon 	1/4
% Cheese, cottage 	Cup 	8
% Cheese, cream 	Cup 	8
% Cheese, grated 	Cup 	4
% Cherries, glaceed 	Cup 	6 1/2
% Chicken, cooked, cubed 	Cup 	5 1/3
% Chili powder 	Tablespoon 	1/4
% Chili Sauce 	Cup 	11 1/4
% Chocolate, grated 	Cup 	4 1/2
% Chocolate, melted 	Cup 	8
% Cinnamon, ground 	Tablespoon 	1/4
% Citron, dried, chopped 	Cup 	6 1/2
% Cloves, ground 	Tablespoon 	1/4
% Cloves, whole 	Cup 	3
% Cocoa 	Cup 	4
% Coconut, shredded 	Cup 	2 1/2
% Corn, canned 	Cup 	8
% Corn flakes 	Cup 	1
% Cornmeal 	Cup 	5 1/3
% Corn syrup 	Cup 	12
% Cornstarch 	Tablespoon 	1/4
% Cornstarch 	Cup 	4 1/2
% Cracker crumbs 	Cup 	3
% Cranberries, raw 	Cup 	4
% Cranberries sauce 	Cup 	8
% Cream of tartar 	Tablespoon 	1/3
% Cream of wheat 	Cup 	6
% Cream, whipping 	Cup 	8
% Cream, whipped 	Cup 	4
% Cucumbers, diced 	Cup 	5 1/3
% Currants, dried 	Cup 	5 1/3
% Curry powder 	Tablespoon 	1/4
%
%
%
% Item Volume 	 Ounces  	Food Index
% Dates, pitted 	Cup 	6 1/5
%
%
%
% Item Volume 	 Ounces  	Food Index
% Eggs, dried, whites 	Cup 	3 1/4
% Eggs, dried, yolks 	Cup 	2 3/4
% Eggs, fresh, whites (9) 	Cup 	8
% Eggs, fresh, yolks (10) 	Cup 	8
% Eggs, raw, shelled (5 eggs) 	Cup 	8
%
%
%
% Item Volume 	 Ounces  	Food Index
% Farina, raw 	Cup 	5 1/3
% Figs, dried, chopped 	Cup 	6 1/2
% Flour, all-purpose 	Cup 	4
% Flour, bread, unsifted 	Cup 	4 1/2
% Flour, bread, sifted 	Cup 	4
% Flour, cake/pastry, sifted 	Cup 	3 1/3
% Flour, rye 	Cup 	2 3/4
% Flour, soy 	Cup 	3 1/4
% Flour, wheat 	Cup 	4 1/4
%
%
%
% Item Volume 	 Ounces  	Food Index
% Gelatin, granulated 	Tablespoon 	1/4
% Gelatin, granulated 	Cup 	5 1/3
% Ginger, ground 	Tablespoon 	1/5
% Ginger, ground 	Cup 	3 1/4
% Grapes, cut, seeded 	Cup 	5 3/4
% Grapes, whole 	Cup 	4
%
%
%
% Item Volume 	 Ounces  	Food Index
% Ham, cooked, diced 	Cup 	5 1/3
% Honey 	Cup 	12
% Horseradish 	Tablespoon 	1/2
%
%
%
% Item Volume 	 Ounces  	Food Index
% Jam 	Cup 	12
% Jelly 	Cup 	10 2/3
%
%
%
% Item Volume 	 Ounces  	Food Index
% Lard 	Cup 	8
% Lettuce, shredded 	Cup 	2 1/4
%
%
%
% Item Volume 	 Ounces  	Food Index
% Margarine 	Cup 	8
% Marshmallows, large  	80 each 	16
% Mayonnaise 	Cup 	8
% Meat, cooked, chopped 	Cup 	8
% Milk, liquid 	Cup 	8 1/2
% Milk, condensed 	Cup 	10 2/3
% Milk, evaporated 	Cup 	9
% Milk, nonfat dry 	Cup 	4
% Milk, nonfat dry 	Tablespoon 	1/4
% Mincemeat 	Cup 	8
% Molasses 	Cup 	12
% Mustard, dry, ground 	Cup 	3 1/2
% Mustard, prepared 	Tablespoon 	1/2
% Mustard seed 	Tablespoon 	2/5
%
%
%
% Item Volume 	 Ounces  	Food Index
% Noodles, cooked 	Cup 	5 1/3
% Nutmeats 	Cup 	4 1/2
% Nutmeg, ground 	Tablespoon 	1/4
%
%
%
% Item Volume 	 Ounces  	Food Index
% Oil, vegetable 	Cup 	8
% Onions, chopped 	Cup 	6 1/2
% Oysters, shucked 	Cup 	8
%
%
%
% Item Volume 	 Ounces  	Food Index
% Paprika 	Tablespoon 	1/4
% Parsley, coarsely chopped 	Cup 	1
% Peanuts 	Cup 	5
% Peanut Butter 	Cup 	9
% Peaches, chopped 	Cup 	8
% Peas, canned, drained 	Cup 	8
% Peas, dried, split 	Cup 	6 3/4
% Pears, canned, drained, diced 	Cup 	6 1/2
% Pecans 	Cup 	4 1/2
% Pepper, ground 	Tablespoon 	1/4
% Pepper, ground 	Cup 	4
% Peppers, green, chopped 	Cup 	5 1/3
% Pimiento, chopped 	Cup 	6 1/2
% Pineapple, crushed 	Cup 	8
% Poppy seed 	Cup 	5
% Potatoes, cooked, diced, mashed 	Cup 	8
% Potato chips 	Cup 	1
% Prunes, dried 	Cup 	6 1/2
% Prunes, cooked, pitted 	Cup 	5
% Pumpkin, cooked 	Cup 	6 1/2
%
%
% Item Volume 	 Ounces  	Food Index
% Raisins 	Cup 	5 1/3
% Raisins, after cooking 	Cup 	7
% Raspberries 	Cup 	4 3/4
% Rhubarb, cooked 	Cup 	6 1/2
% Rhubarb, raw, 1" diced 	Cup 	4
% Rice, uncooked 	Cup 	8
% Rice, cooked 	Cup 	8 1/2
% Rice, puffed 	Cup 	3/5
% Rutabaga, cubed 	Cup 	4 3/4
%
%
% Item Volume 	 Ounces  	Food Index
% Sage, ground 	Cup 	2
% Salad dressing 	Cup 	8
% Salmon, canned 	Cup 	8
% Salt 	Tablespoon 	2/3
% Sauerkraut 	Cup 	5 1/3
% Sesame seed 	Tablespoon 	1/3
% Sesame seed 	Cup 	5 3/8
% Shallots, diced 	Tablespoon 	2/5
% Shortening 	Cup 	7
% Soda, baking 	Tablespoon 	2/5
% Soybeans 	Cup 	7
% Spinach, raw 	Quart 	3 3/4
% Spinach, cooked 	Cup 	8
% Squash, Hubbard, cooked 	Cup 	8
% Strawberries 	Cup 	7
% Suet, ground 	Cup 	4 1/2
% Sugar, brown, lightly packed 	Cup 	5 1/3
% Sugar, brown, solidly packed 	Cup 	8
% Sugar, granulated 	Cup 	8
% Sugar, powdered, sifted 	Cup  	5 1/3
%
%
% Item Volume 	 Ounces  	Food Index
% Tapioca, quick-cooking 	Cup 	5 1/3
% Tapioca, pearl 	Cup 	5 3/4
% Tea, loose-leaf 	Cup  	2 2/3
% Tea, instant 	Cup 	2
% Tomatoes, canned 	Cup 	8
% Tomatoes, fresh, diced 	Cup 	7
% Tuna 	Cup 	8
%
%
% Item Volume 	 Ounces  	Food Index
% Vanilla 	Tablespoon 	1/2
% Vinegar 	Cup 	8
%
%
% Item Volume 	 Ounces  	Food Index
% Walnuts, shelled 	Cup 	4
% Water 	Cup 	8
%
%
% Item Volume 	 Ounces  	Food Index
% Yeast, compressed cake 	each 	3/5
% Yeast, envelope 	each 	1/4
%
%
%
%
% Pasteurization time:
% 130°F 	121.0 minutes
% 135°F 	38.3 minutes
% 140°F 	12.1 minutes
% 145°F 	3.8 minutes
% 150°F 	1.2 minutes
% 155°F 	23.0 seconds
% 160°F 	7.2 seconds
% 165°F 	2.3 seconds
%
% Cookies: almost universally 190\degF (internal)
%
% BEEF:
% Rare: 125\degF, 52\degC
% Medium Rare: 135\degF, 57\degC
% Medium: 145\degF, 63\degC
% Medium Well: 155\degF, 68\degC
% Well Done: 165\degF, 74\degC
%
% As a rule of thumb, small pieces of meat like steaks, chops, and chicken
% will not continue to cook much after you take them off the heat, certainly
% less than 5°F. But large thick roasts of beef, lamb, veal, pork loin, or
% even large turkey breasts should come out of the heat at 5°F less than the
% desired temp and they will rise about 5°F in 10 minutes of resting. If you
% are cooking at high temps, carryover can be up to 10°F. If you are cooking
% at low temps, it might not go up the full 5°F. But 5°F is a good rule of
% thumb for most roasts.
%
% Today trichinosis has, for all practical purposes, been eradicated in
% developed countries. The annual average is now fewer than 11 cases per year
% in the US, most associated with eating undercooked wild game such as bear.
%
% Trichinosis is killed at 138°F
%
% Researchers tell us that a significant percent of chickens and turkeys have
% salmonella in their juices. At 165°F white meat is still slightly juicy.
% Much higher and you'll be eating cardboard. The transition happens rapidly.
% The juices should run clear and any pink could be dangerous. That said,
% safe chicken can have some bright red parts attached to the bone because
% they are being brought to slaughter younger and before the bones are fully
% hard (Their bones have not yet matured and are still somewhat soft and
% porous. As a result, there can be seepage of bone marrow through the soft
% bone and into the surrounding meat.).
%
% \begin{tabular}{llll}
%   Meat                     & USDA Min & Recommended  & Notes\\
%   Beef                     & 145\degF & 130-135\degF & medium rare\\
%   Lamb                     & 145\degF & 140-145\degF & touch of pink; translucent pink juice\\
%   Pork                     & 145\degF & 135-145\degF & cream colored, some pink\\
%   Fish                     & 145\degF & 130-145\degF & slightly translucent\\
%   Chicken                  & 165\degF & 165\degF     & clear juices\\
%   Ground Meats             & 160\degF & 160\degF     & \\
%   Pre-cooked ham, sausage  & 140\degF & 140\degF     & \\
%   High collagen meats      & 145\degF & 190-205\degF & \\
%   Cassaroles and leftovers & 165\degF & \\
% \end{tabular}
%
% Covered vs uncovered: Temp difference of approx ____
%
%
%
%
%
% % \setlength{\columnseprule}{.4pt}
% % \begin{multicols}{3}
% % \begin{center}
% %   {\footnotesize
% %     \begin{tabular}{r@{$\;=\;$}l}
% %       1 apothecary ounce & 8 drams\\
% %       1 apothecary pound & 12 ounces\\
% %       1 barrel of liquid & 31 gallons\\
% %       1 barrel of petroleum & 42 gallons\\
% %       1 bushel & 4 pecks\\
% %       1 carat & 3.086 grains\\
% %       1 cord & 128 ft$^3$\\
% %       1 cup & 8 fluid ounces\\
% %       1 dram & 3 scruples\\
% %       1 dry pint & 33.60 in$^3$\\
% %       1 dry quart & 2 dry pints\\
% %       1 fathom & 6 feet\\
% %       1 fluid ounce & 2 tablespoons\\
% %       1 foot & 12 inches\\
% %     \end{tabular}
% %     \begin{tabular}{r@{$\;=\;$}l}
% %       1 furlong & 40 rods\\
% %       1 gallon & 3.785 liters\\
% %       1 gallon & 4 quarts\\
% %       1 grain & 0.0648 grams\\
% %       1 inch & 2.540 centimeters\\
% %       1 land league & 3 miles\\
% %       1 long ton & 2240 pounds\\
% %       1 marine league & 3 nautical miles\\
% %       1 meter & 3.2808 feet\\
% %       1 mile & 320 rods\\
% %       1 mile & 5280 feet\\
% %       1 nautical mile & 1.15 miles\\
% %       1 ounce & 437.5 grains\\
% %     \end{tabular}
% %     \begin{tabular}{r@{$\;=\;$}l}
% %       1 peck & 8 dry quarts\\
% %       1 pennyweight & 24 grains\\
% %       1 pint & 16 fluid ounces\\
% %       1 pound & 0.4536 kilograms\\
% %       1 pound & 16 ounces\\
% %       1 quart & 2 pints\\
% %       1 rod & 5.5 yards\\
% %       1 scruple & 20 grains\\
% %       1 tablespoon & 3 teaspoon\\
% %       1 ton & 2000 pounds\\
% %       1 troy ounce & 480 grains\\
% %       1 troy pound & 12 troy ounces\\
% %       1 yard & 3 feet\\
% %     \end{tabular}}
% % \end{center}
% % \end{multicols}


\subsection{Slow-Cooker / Crock-Pot Cooking}
Crock-Pots ``Low'' and ``High'' settings aren't about temperature, they are
about the time it takes to get to 209\degF. A slow cooker will start at
140\degF, then ease up to full temperature --- about 3--4 hours for
``High'' and about 8 hours for ``Low''. They also will be more accurate and
have less variation than an oven. Something like:

\bigskip

\begin{center}
  \psset{xunit=.4in,yunit=0.01,labelsep=3.5pt}
  \begin{pspicture}(8,212)
    \psline(0,0)(8,0)\psline(0,0)(0,212)
    \uput[180](0,209){\footnotesize 209\degF}
    \uput[-90](8,0){\footnotesize Cook Time}
    \psline(0,0)(1,104)(2,140)(3,162)(4,185)(6,200)(8,209)
    \uput{3pt}[125](1,104){\footnotesize \vfrac18: 100\degF}
    \uput[135](2,140){\footnotesize \vfrac28: 140\degF}
    \uput[130](3,162){\footnotesize \vfrac38: 160\degF}
    \uput[110](4,185){\footnotesize \vfrac48: 185\degF}
    \uput[135](6,200){\footnotesize \vfrac68: 200\degF}
    \uput[110](8,209){\footnotesize \vfrac88: 209\degF}
  \end{pspicture}
\end{center}

\medskip

Unfortunately, you can't just use an oven and manually hit these
temperatures because a crock pot has direct contact so the food hits these
targets much faster. Thus, if you need a crock pot, borrow one.
% The crock pot coocks via conduction into the contained liquid while the
% oven is cooking via air convection.


%%% Local Variables:
%%% mode: latex
%%% TeX-master: "../Cookbook"
%%% End:
