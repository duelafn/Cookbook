
\section{Thickeners}
From: \url{https://www.finecooking.com/article/thickeners}

All starches begin to thicken around 140\degF. But to achieve full
thickening power, flour and cornstarch, built of \textsl{amylose}, must
come all the way to a boil (and be held just below the boiling point for
several minutes to cook off starch flavor). Root starches like tapioca,
built of \textsl{amylopectin}, thicken well before the boiling point.

Prolonged cooking, stirring, and exposure to acids like lemon juice, wine,
and vinegar weaken all starches' thickening power. Different starches,
however, can endure different amounts of heat, agitation, and acidity
before they start to break down.

To prevent lumps, always stir the starch into a small of amount of cool
water, then stir this starch slurry into the hot liquid you want to
thicken.

\subsection{Corn starch}
Corn starch can withstand a good amount of cooking and stirring before it
begins to break down. Frequently used for thickening pastry cream, pie
fillings, and puddings. Also great for delicate sauces and gravies that you
want to be translucent, like stir-fry sauces. Simmer for at least a minute
to eliminate the pasty flavor of raw starch.

\subsection{Flour}
Flour contains protein and other components in addition to starch, so it
has only about half the thickening power of other starches. Proteins make
flour-thickened sauces look cloudy. Flour works best in white sauces, pan
gravies, beef stew, and apple or pear pies.

To achieve full thickening power and eliminate raw flour taste,
flour-thickened mixtures must be brought to a boil and then cooked for
about 3 minutes. Because flour thickens more as it cools, stop cooking
gravies and sauces when they're a bit thinner than their ideal consistency.

A common way to get rid of the flour taste is to first form a roux by
mixing flour with oil or fat and cooking the paste for a few minutes before
mixing in the liquids. For stews, dredging stew meat in flour before
browning it thicken the stew later.

\subsection{Tapioca} Tapioca-thickened fillings are clear and have a more
jelly-like consistency than those thickened with other starches. Instant
tapioca granules don't completely dissolve; they may linger in pie fillings
as soft, clear beads. It thickens juices faster than flour or cornstarch,
so tapioca is great for fruit pies which throw off a lot of juice (e.g.,
berry and peach). Tapioca is robust to freezing and won't weep liquid when
thawed.

Don't use instant tapioca for pies with open lattices or large steam vents
because the granules will be exposed directly to the hot air of the oven
and won't dissolve. It's also not ideal for pan sauces or stovetop custards
because it can't withstand a lot of stirring and boiling.

For best results, let pearl tapioca sit with the fruit for 5 to 10 minutes
before you bake the pie so that the fruit juices can begin to soften the
granules. And before you remove a pie from the oven, make sure the juices
at the center are bubbling, even if it seems the juices at the edge have
been fully cooked for quite a while.

\subsection{Pectin} Regular pectin requires the presence of sugar and
acid in order to gel unless you use a ``no-sugar pectin''.

\subsection{Gelatin} Must bloom in at least a few tablespoons of cold liquid per
packet (\fourth \oz) of gelatin for 5--10 minutes. Gelatin is hydrolyzed
collagen\footnote{Hydrolysis tends to break down into individual amino acids where
  denaturation is more destructive and breaks down the protein structure and
  function.}, a specific protein. Collagen in its natural form does not have the
thickening properties that gelatin has.

\subsection{Agar-agar} vegan alternative for gelatin, but requires
longer soak and longer cook time.

%%% Local Variables:
%%% mode: latex
%%% TeX-master: "../Cookbook"
%%% End:
