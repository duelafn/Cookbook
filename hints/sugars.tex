
\section{Sugars}
\subsection{Fructose}
\begin{itemize}
\item Fructose caramelizes faster than sucrose, so bake up to 25\degF lower
  than the recipe instructs when substituting with it.
% \item Fructose overloads your liver by working it to turn fructose directly to fat (which is then stored in blood).
% \item Glucose overloads your pancreas producing insulin since glucose goes directly to blood.
\end{itemize}

% \subsection{Molasses}
% Wikipedia ``Molasses'', 2006-03-17
%
% Molasses or treacle is a thick, syrupy derivative of the juice of the
% sugarcane plant or the processing of sugar beet. The word molasses comes
% from the Portuguese word mela\c co. The quality of molasses depends on the
% maturity of the sugar cane or beet, the amount of sugar extracted, and the
% method of extraction.
%
% \paragraph{Cane Molasses}
% \begin{itemize}
% \item Sulphured molasses is made from green (not yellow) sugar cane and is
%   treated with sulphur fumes during the sugar extraction process.
% \item Each season, the sugar cane plant is harvested and stripped of its
%   leaves. Its juice is then extracted from the canes (usually by crushing
%   or mashing), boiled until it has reached the appropriate consistency, and
%   processed to extract the sugar. The results of this first boiling and
%   processing is first molasses, which has the highest sugar content because
%   comparatively little sugar has been extracted from the juice.
% \item Second molasses is created from a second boiling and sugar
%   extraction, and has a slight bitter tinge to its taste. Further rounds of
%   processing and boiling yield the dark blackstrap molasses, which is the
%   most nutritionally valuable, and thus often sold as a health supplement,
%   as well as being used in the manufacture of cattle feed, and for other
%   industrial uses.
% \end{itemize}
% Molasses can also be produced from several grasses such as sorghum.
%
% \paragraph{Sugar Beet Molasses}
% Molasses from the sugar beet is different from cane molasses. Only the
% syrup left from the final crystallisation stage is called molasses;
% intermediate syrups are referred to as high green and low green. It is
% about 50\% sugar by dry weight, predominantly sucrose but also containing
% significant amounts of glucose and fructose. The non-sugar content includes
% many salts such as calcium, potassium, oxalate and chloride. As such, it is
% unpalatable and is mainly used as an additive to animal feed or as a
% fermentation feedstock.
%
% It is possible to extract additional sugar from beet molasses through a
% process known as molasses desugarisation. This technique exploits
% industrial scale chromatography to separate sucrose from non sugar
% components. The technique is only economically viable in areas where the
% price of sugar is supported above the world market --- e.g., in areas with
% trade barriers. It is prevalent in the United States and is also seen
% within the European Union.
%
% \paragraph{Substitutions}
% Molasses is a common ingredient in baking, often used in baked goods such
% as gingerbread cookies. In such recipes, it is possible to replace each cup
% of molasses with one of the following:
% \begin{itemize}\tightitems
% \item 1 cup honey
% \item 3/4 cup firmly packed brown sugar
% \item 1 cup dark corn syrup
% \item 1 cup pure maple syrup
% \end{itemize}
%
% \paragraph{Other information}
% \begin{itemize}
% \item Molasses is a chelating agent. An object coated with iron rust placed
%   for two weeks in a mixture of one part molasses to nine parts water will
%   lose its rust due to the chelating action of the molasses.
% \item A famous incident involving molasses was the Boston Molasses Disaster
%   on January 15, 1919, in which a large molasses storage tank burst and
%   flooded a neighborhood of Boston, killing 21 and injuring 150.
% \item The British pudding Treacle Tart does not use any treacle but golden syrup.
% \item Molasses can be fermented into rum.
% \end{itemize}
%
%
% \subsection{Brown Sugar}
% Wikipedia ``Brown sugar'', 2006-03-17
%
% Brown sugar is an unrefined or partially refined soft sugar consisting of
% sugar crystals combined with molasses. Brown sugar is produced similarly to
% white sugar, with two exceptions. Its crystals are left much smaller than
% for white sugar, and the syrup or molasses is not washed off completely.
% Brown sugar contains from 3.5\% molasses (light brown sugar) to 6.5\%
% molasses (dark brown sugar).
%
% Many brown sugar producers produce brown sugar by adding molasses to
% completely refined white sugar crystals in order to more carefully control
% the ratio of molasses to sugar crystals, and to reduce manufacturing costs.
% Brown sugar prepared in this manner is often much coarser than its
% unrefined equivalent, and its molasses may be easily separated from the
% crystals to yield white sugar (which is not possible with unrefined sugar).
% This is mainly done for inventory control and convenience.
%
% Brown sugar can be made at home by mixing white granulated sugar with
% molasses, using one tablespoon of molasses for every cup of white sugar
% (one-sixteenth or 6.25\% of the total volume). Thorough blending will yield
% dark brown sugar; for light brown sugar, between one and two teaspoons of
% molasses per cup should be used instead. It is, however, simpler to
% substitute molasses for an equal portion of white sugar while cooking,
% without mixing them separately.
%

%%% Local Variables:
%%% mode: latex
%%% TeX-master: "../Cookbook"
%%% End:
