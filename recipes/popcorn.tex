
\begin{recipe}{Popcorn}
  \rating    5
  \nutrition 3
  \health    2
  \workload{light}
  \time{15}
  \yield{10~cups}
  \servings{2}
  \source{}
  % \photo{}
  \maketitle

  \begin{ingredients2}
    \half \cup & poppcorn\\
    3 \T & butter (melted)\\
    1.5 pinch & salt
  \end{ingredients2}

  If using an air popper, start pouring butter while still popping for even
  coverage.
\end{recipe}

\begin{recipe}{Popcorn Balls}
  \rating    5
  \nutrition 3
  \health    2
  \workload{light}
  \time{15}
  \yield{10~cups}
  \servings{2}
  \source{}
  % \photo{}
  \maketitle

  \begin{ingredients2}
    1 \cup & poppcorn\\
    3 \T & water\\
    \half \cup & sugar\\
    \half \cup & corn syrup\\
    2 \T & butter\\
    1 \t & vanilla\\
    pinch & salt
  \end{ingredients2}

  % Tried 240° was quite sticky.
  %
  % Original recipe called for twice the goop to popcorn ratio but I misread it. The
  % lower ratio is pretty good and might be better (or at worst increase by 50%).
  %
  Pop popcorn. Heat sugars and water in small sauce pan to 260\degF. Once temperature
  reached, stir in butter, vanilla, and salt. Drizzle over popcorn as someone else
  shakes in a l;arge bowl or pan.
\end{recipe}

%%% Local Variables:
%%% mode: latex
%%% TeX-master: "../Cookbook"
%%% End:
