
\begin{recipe}{Whipped Cream}%\tag{}
  \rating    5
  \nutrition 1
  \health    1
  \workload{light}
  \time{10}
  \yield{1 pie}
  \servings{8}
  \maketitle
  % \photo{}

  \begin{ingredients2}
    2 \cups     & whipping cream\\
    \third \cup & powdered sugar\\
    1--2 \t     & vanilla
  \end{ingredients2}

  Combine and whip

  \section{Stabilized Whipped Cream}
  Using gelatin produces a stable whipped cream with a pure unaltered taste.
  \begin{ingredients2}
    1 packet & unflavored gelatin
  \end{ingredients2}
  Bloom gelatin in a few tablespoons of water, 5 mins. Microwave for 5--10
  seconds to melt gelatin. Stir in a 2~\t cream to temper.

  Whip cream and sugar as above. At soft peaks, slow down mixer slightly
  (to 75\%) and slowly drizzle in the liquid gelatin. Must keep mixing
  because gelatin solidifies quickly. Increase speed and finish whipping to
  stiff peaks. Warning: It is rumored that the combined mixture can whip to
  butter more easily than the whipped cream alone.

  \section{Stabilized Whipped Cream}
  Using pudding alters the flavor quite a bit, but is tasty.
  \begin{ingredients2}
    2 \T & instant pudding
  \end{ingredients2}
  Whip cream and (less) sugar as above. At soft peaks, slow down mixer
  slightly (to 75\%) and sprinkle instant pudding. Increase speed and
  finish whipping to stiff peaks.
\end{recipe}

%%% Local Variables:
%%% mode: latex
%%% TeX-master: "../Cookbook"
%%% End:
