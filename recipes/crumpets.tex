
\begin{recipe}{Crumpets}
  % \tag{}
  \rating    0
  \nutrition 3
  \health    3
  \workload{medium}
  \time{100}
  \source{Internet: ``Miss Libby'' on answers.yahoo.com}% http://answers.yahoo.com/question/index?qid=20070831094047AAXLgoT
  % \photo{}
  \maketitle

  \begin{ingredients2}
    450 \g  & white flour (3 \cups)\\
    10 \g   & active dry yeast (2 \t)\\
    2\threefourth \cups & warm water\\
    1\half \t          & bread improver (optional)\\
    1\half \t          & salt\\
    1 \t               & sugar or honey\\
    2 \T               & powdered milk\\
    1 \t               & bicarbonate of soda\\
    2 \T               & warm water
  \end{ingredients2}

  Combine yeast, sugar and one~cup of warm water into a mixng bowl. Stand
  in a warm place until yeast is dissolved and happy. Sift flour, milk and
  \emph{salt} (for yeast) into another bowl. Add yeasty liquid and, by
  hand, mix to a thick batter, cover, and stand in a warm place until well
  risen and bubbly. This will take about an hour.

  Combine the bicarbonate of soda and the extra water, and add this mix to
  the dough. MIX WELL. Then leave this mixture to stand, covered, in a warm
  place for a further 15 minutes.

  Preheat a heavy frypan to a low-medium heat. Coat the crumpet rings with
  spray oil. Fill rings almost to the top (but allow some room to rise).
  Cook for 4--8 minutes, until bubbles appear over the entire surface, and
  the dough appears `dry'.

  Remove the ring, turn the crumpet over and cook for a further 30--60
  seconds to brown the top. Remove from the pan and cool on a cake rack.

  Toast the crumpets and serve with lashings of cream and .... ENJOY.

  \begin{note}
    We didn't use bread improver or powdered milk. We also just cooked them
    as ``tiny pancakes'' without using crumpet rings. Result was reported
    as ``as good as mom's''.
  \end{note}
\end{recipe}

%%% Local Variables:
%%% mode: latex
%%% TeX-master: "../Cookbook"
%%% End:
