
\begin{recipe}{Coffee}%\tag{}
  \maketitle

  All measurements can apply to beans before grinding or to the grounds,
  they measure more or less the same volume.

  \textsl{Grinding:} For a standard hand-held grinder, grind about
  \half~cup of beans at a time for 10 seconds. Give the grinder a shake at
  about the 5 second mark, but there is no need to pulse the grinder.

  \section{Pour-over}
  Makes the best coffee. Use 2~\T grounds for each {8~\oz} {cup} of coffee.
  Add grounds to filter in filter holder (dripper), boil water, pour over
  the grounds. Coffee is done after a single pass. Ideally, dripper should
  be sized so that total brew time is about 5 minutes (otherwise you are
  brewing too much/too little for the dripper).

  \section{Drip Coffee-maker}
  Use 1--2~\T grounds for each {8~\oz} {cup} of coffee (sometimes 2~\T per
  cup won't fit in basket). Note that markings on carafe are for 6~\oz
  cups, so the standard ``12-cup'' carafe really makes only 8~cups of
  coffee (thus a full pot uses \half to 1~cup grounds). Put filter and
  grounds in basket, pour cold water into back reservoir and turn on. Takes
  about 10--15 minutes to brew.

  \section{Percolator Coffee}
  Typically only for large quantities of coffee. Coffee percolates up the
  tup to the top. When all water is hot it just start boiling rather than
  percolating. Automatic makers will generally turn on a light and reduce
  heat at this point to stop the brewing process and keep the coffee warm.

  Use 1--1\half~\T grounds for each {8~\oz} {cup} of coffee (long brew time
  makes it stronger/bitterer). Assemble maker, there will be a filter below
  the grounds and a water spreader above the grounds. Takes 30--60 minutes
  to brew.

\end{recipe}

%%% Local Variables:
%%% mode: latex
%%% TeX-master: "../Cookbook"
%%% End:
