
\begin{recipe}{Grilled Sausages}
  \rating    0
  \nutrition 0
  \health    0
  \workload{}
  \time{}
  \yield{}
  \servings{}
  \source{}
  \photo{grilled-sausages}
  \maketitle

  \begin{ingredients2}
    2 & sausages per person\\
    & cheese\\
    & lettuce\\
    & green pepper\\
    & cucumber\\
    & onion\\
    & tomato\\
    & condiments\\
    & buns\\
    & hamburgers (opt)\\
    & potatoes (opt)\\
    & zucchini (opt)
  \end{ingredients2}
\end{recipe}

\begin{recipe}{Hamburgers}\tag{beef}
  \rating    0
  \nutrition 0
  \health    0
  \workload{}
  \time{}
  \yield{}
  \servings{}
  \source{}
  \photo{hamburgers}
  \maketitle

  \begin{ingredients2}
    \half \lb & meat per person\\
    & cheddar cheese\\
    & lettuce\\
    & green pepper\\
    & cucumber\\
    & onion\\
    & tomato\\
    & condiments\\
    & hamburger buns\\
    & sausages (opt)\\
    & potatoes (opt)\\
    & zucchini (opt)
  \end{ingredients2}

  I like 90\% lean and weigh patties out to \fourth~\lb and flatten very thin (about
  6\inch diameter) to maximize flavor and so that they fit buns after shrinking.
\end{recipe}

%%% Local Variables:
%%% mode: latex
%%% TeX-master: "../Cookbook"
%%% End:
