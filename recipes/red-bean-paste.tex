
\begin{recipe}{Red Bean Paste (Anko)}%\tag{}
  \rating    3
  \nutrition 3
  \health    3
  \workload{medium}
  \time{120}
  \yield{3~cups}
  \servings{}
  \source{\website{Omnivore's Cookbook}{https://omnivorescookbook.com/recipes/how-to-make-red-bean-paste}}
  \maketitle
  % \photo{}

  \begin{ingredients2}
    300 \g & red beans\\
    200--250 \g & sugar\\
    \eigth \t & salt
  \end{ingredients2}

  Rinse and soak beans overnight.

  Dump water, add beans and 4~cups water to saucepan. Bring to boil, then
  reduce to low. Simmer 1 hour until soft and mashable.

  Add sugar and increase to medium-high heat. Stir constantly until very
  thick, we are making a paste! Note: Reduce temperature as sauce thickens.

  Transfer from pan to encourage cooling.
\end{recipe}


\begin{recipe}{Sticky rice cake with red bean paste}%\tag{}
  \rating    3
  \nutrition 3
  \health    3
  \workload{medium}
  \time{60}
  \yield{12 cakes}
  \servings{12}
  \source{\website{Omnivore's Cookbook}{https://omnivorescookbook.com/recipes/sticky-rice-cake-with-red-bean-paste}}
  \maketitle
  % \photo{}

  \begin{ingredients2}
    250 \g & glutinous rice flour\\
    1\half \cups & red bean paste\\
    1 \T & vegetable oil\\
    \half \cup & roasted sesame seeds
  \end{ingredients2}

  First prepare bean paste. Note: It is not useful to make dough ahead of
  time, it will dry out (through absorption) and need to be re-kneaded
  anyway. Wait until bean paste is done before starting this dough.

  In 4--5 stages, incorporate up to 1~cup of water into the rice flour.
  dough should be soft and easy to shape. Knead until uniform. Roll and
  divide into 12.

  Flatten each ball to about \quarter\inch~thick, 2\half~\inch diameter.
  Put 2~\T bean paste in and pinch closed (seal isn't terribly important).
  roll in sesame.

  Fry them on low in a heavy pan with a little oil. Don't cook too quickly
  or the dough will not cook all the way through. Dump out all loose sesame
  seeds between each batch, else they will burn. Serve warm.
\end{recipe}


\begin{recipe}{Daifuku}%\tag{}
  \rating    3
  \nutrition 3
  \health    3
  \workload{medium}
  \time{60}
  \yield{12}
  \servings{12}
  \source{\website{Just One Cookbook}{https://www.justonecookbook.com/daifuku}}
  \maketitle
  % \photo{}

  \begin{ingredients2}
    100 \g & shiratamako (glutinous rice flour)\\
    \threefourth \cup & water\\
    50 \g & sugar\\
    & corn or tapioca starch\\
    1\half \cup & red bean paste (anko)
  \end{ingredients2}

  Mix sugar and rice flour, then add water. Steam in the bowl for 15
  minutes, stirring half way. Place towl around lid of steamer to prevent
  drips from landing in the dough.

  sprinkle starch on a cutting board and dump doung on top. pat out to
  \half\inch thick and let cool. Once mostly cool, flip out onto
  starch-covered wax paper and roll out thin.

  Cool 15 minutes in the refrigerator, then cut into 4\half\inch rounds.
  Fill, pinch, and seal then store in a cool dry place.
\end{recipe}


%%% Local Variables:
%%% mode: latex
%%% TeX-master: "../Cookbook"
%%% End:
