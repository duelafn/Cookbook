
\begin{recipe}{Baklava}
  % \tag{}
  \rating    0
  \nutrition 2
  \health    1
  \workload{medium}
  \time{30}
  \yield{loaf pan (4 pieces)}
  \servings  4
  \source{St.~Andrews Greek Festival}
  \photo{baklava}
  \maketitle

  \section{Baklava}
  \begin{ingredients2}
    \fourth \lb & \wrap{walnuts or pecans, finely chopped}\\
    4 \t & sugar\\
    1 \t & ground cinnamon\\
    \half \t & nutmeg\\
    10 sheets & phyllo\\
    \fourth \lb & \wrap{butter clarified, melted and warm}
  \end{ingredients2}

  Thaw phyllo overnight.

  Combine nuts, sugar and spices. Trim Phyllo to pan size (use a loaf pan).
  Place 6 Phyllo sheets into buttered pan, brushing each sheet generously
  with butter. Sprinkle a thin layer of nut mixture evenly over the Phyllo.
  Cover with 4 buttered Phyllo sheets. Continue alternating Phyllo, butter
  and nut mixture until all nuts are used. Top with 8 buttered Phyllo
  sheets. Cut into triangles or squares. lf desired, stud each piece with a
  whole clove.

  Bake at 325\degF for 30 minutes, lower oven to 300\degF and bake 30
  minutes longer or until golden brown. Pour warm syrup over hot baklava.
  Let stand for several hours or over night. Store in airtight containers
  in a cool place; it will keep for several weeks. Freezes well; thaw to
  room temperature and serve.

  \section{Syrup}
  \begin{ingredients2}
    \fourth \cup & water\\
    \half \cup & sugar\\
    \half \T & lemon juice\\
    \half \T & honey
  \end{ingredients2}

  Boil sugar, water and lemon juice 10 minutes, stirring at first to
  dissolve sugar. Stir in honey, simmer 1 minute.

  \begin{note}
    Uses remainder of phyllo after making \reciperef{Spanakopita}.
  \end{note}
\end{recipe}

%%% Local Variables:
%%% mode: latex
%%% TeX-master: "../Cookbook"
%%% End:
