
\begin{recipe}{Grasshopper Bars}\tag{cookies}
  \rating    0
  \nutrition 1
  \health    1
  \workload{heavy}
  \time{180}
  \yield{24}
  \servings{24}
  \source{seriouseats.com}
  % \photo{}
  \maketitle

  \section{Brownie Base}
  \begin{ingredients2}
    \threefourths \cups & flour\\
    \half \t & salt\\
    1 \T & cocoa powder\\
    5 \oz & dark chocolate\\
    \half \cup & unsalted butter\\
    \threefourths \cups & sugar\\
    \fourth \cup & light brown sugar\\
    3 & large eggs\\
    1 \t & vanilla extract
  \end{ingredients2}

  Let the eggs sit out so that they are at room temperature. Preheat oven
  to 325\degF. Butter the sides and bottom of a 9 by 13 inch pan. Line the
  bottom with a sheet of parchment paper, and butter the parchment. In a
  medium bowl, whisk together the flour, salt, and cocoa powder.

  Set up a large double boiler. Coarsely chop the dark chocolate and
  butter. Place the chocolate and the butter in the bowl of the double
  boiler and stir occasionally until the chocolate and butter are
  completely melted and combined. Turn off the heat, but keep the bowl over
  the water of the double boiler while adding both sugars. Whisk the sugars
  until completely combined. Remove the bowl from the pan. The mixture
  should be at room temperature.

  Add three eggs to the chocolate/butter mixture and whisk until just
  combined. Add the vanilla and stir until combined. Do not overbeat the
  batter at this stage or your brownies will be cakey.

  Sprinkle the flour/cocoa/salt mix over the chocolate. Using a spatula (do
  not use a whisk) fold the dry ingredients into the wet until there is
  just a trace amount of the flour/cocoa mix visible.

  Pour the batter into the prepared pan, smooth the top with an offset
  spatula, and bake for approximately 12 to 15 minutes, rotating halfway
  through the baking time. The brownies should be just a tad underdone (not
  too gooey, but ideally, just 1 minute from being cooked through
  completely). A toothpick inserted into the brownies at an angle should
  contain a few loose crumbs. Remove the brownies from the oven and let
  cool completely while you make the creme de menthe filling.

  \section{Buttercream}
  \begin{ingredients2}
    \threefourths \cups & sugar\\
    2 \T & flour\\
    \threefourths \cup & milk\\
    2 \T & heavy cream\\
    \threefourths \cup & unsalted butter\\
    3 \T & creme de menthe\\
    1 \t & peppermint extract
  \end{ingredients2}

  In a medium heavy-bottomed saucepan, whisk the sugar and flour together.
  Add the milk and cream and cook over medium heat, whisking occasionally
  until mixture comes to a boil and has thickened, 5 to 7 minutes.

  Cut the softened but cool butter into small cubes. Transfer the mixture
  to the bowl of an electric mixer fitted with the paddle attachment. Beat
  on high speed until cool. Reduce the speed to low and add the butter and
  mix until thoroughly incorporated. Increase the speed to medium-high and
  beat until filling is light and fluffy.

  Add the creme de menthe and peppermint extract and mix until combined. If
  the filling is too soft, chill slightly in the refrigerator and then mix
  again until it is the proper consistency.

  If the filling is too firm, place the bowl over a pot of simmering water
  and re-mix to proper consistency. Spread the filling evenly across the
  top of the brownie layer and place the pan in the refrigerator, for a
  minimum of 45 minutes, while you make the chocolate glaze.

  \section{Chocolate Glaze}
  \begin{ingredients2}
    6 \oz & dark chocolate\\
    1 \t & light corn syrup\\
    \half \cup & unsalted butter
  \end{ingredients2}

  Coarsely chop the dark chocolate and the softened butter. In a large
  non-reactive metal bowl, combine the chocolate, corn syrup, and butter.
  Set the bowl over a saucepan of simmering water and cook, stirring with a
  rubber spatula, until the mixture is completely smooth. Remove the bowl
  from the pan and stir vigorously for 1 minute to release excess heat.

  Pour the mixture over the chilled creme de menthe layer and use an offset
  spatula to spread it into an even layer. Place the pan back in the
  refrigerator for 1 hour, or until the glaze hardens.

  Remove the pan from the refrigerator, wait about 15 minutes for the glaze
  to soften slightly, and cut the bars with a warm knife. Cut into squares
  and serve immediately.

  Note: The bars should be stored in the refrigerator.

\end{recipe}

%%% Local Variables:
%%% mode: latex
%%% TeX-master: "../Cookbook"
%%% End:
