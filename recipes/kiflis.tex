
\begin{recipe}{Kiflis}\tag{cookies}
  \rating    0
  \nutrition 1
  \health    1
  \workload{heavy}
  \time*{120}
  \yield{150}
  \servings{150}
  \source{Aunt Gi Horvath}
  % \photo{}
  \maketitle

  \section{Pastry Dough}
  \begin{ingredients2}
    1 \lb & softened butter\\
    6 \cups & sifted flour\\
    12 & eggs (separated)\\
    1 \t & vanilla\\
    1 \cup & sour cream
  \end{ingredients2}

  Make pastry dough the night before, but save the egg whites for the
  filling the next day. To make the dough, work together the butter and
  sifted flour. In a separate bowl, beat slightly the egg yolks, vanilla,
  and sour cream. Work the egg mixture into the flour mixture. Shape the
  flour mix into walnut size balls and chill overnight in refrigerator.

  \section{Walnut Filling}
  \begin{ingredients2}
    2 \lb & ground walnuts\\
    1 \lb & confectioners sugar\\
    1 & grated lemon rind
  \end{ingredients2}

  Beat the egg whites stiff. Mix in the walnuts, confectioners sugar, and
  lemon rind.

  \section{Poppy Seed Filling}
  \begin{ingredients2}
    3 \cups  & poppy seeds (\twothird \lb)\\
    1\fourth \c & sugar\\
    \threefourths \c & milk\\
    \half \c & butter
  \end{ingredients2}

  Combine all ingredients and cook over low heat. Cook 5 minutes until
  slightly thick. If too thick to spread on kifli dough, add more milk.

  \section{Assembling Kiflis}
  Heat oven to 375\degF. Remove dough balls a few at a time and roll paper
  thin on a floured cloth with a flour-covered rolling pin. Sprinkle with
  flour as needed to keep the dough from sticking.

  Place about 1~\T filling in each flattened dough ball. Roll each filled
  flattened dough ball like a jelly roll and curve into a crescent shape.
  Place on cookie sheet.

  Bake kiflis for about 15 minutes, until light brown. Remove from oven,
  cool, then sprinkle with confectioners sugar.
\end{recipe}

%%% Local Variables:
%%% mode: latex
%%% TeX-master: "../Cookbook"
%%% End:
